\chapter{List, Hashtable, Stack, Sort}

\section{Add Two Numbers} %%%%%%%%%%%%%%%%%%%%%%



\subsubsection{Description}
You are given two non-empty linked lists representing two non-negative integers. The digits are stored in reverse order and each of their nodes contain a single digit. Add the two numbers and return it as a linked list.

You may assume the two numbers do not contain any leading zero, except the number 0 itself.

\textbf{Input:} \code{(2 -> 4 -> 3) + (5 -> 6 -> 4)}

\textbf{Output:} \code{7 -> 0 -> 8}

\subsubsection{Solution}

\begin{Code}
public ListNode addTwoNumbers(ListNode l1, ListNode l2) {
    ListNode dummy = new ListNode(0), cur = dummy;

    for (int carry = 0; l1 != null || l2 != null || carry > 0; ) {
        int n1 = l1 != null ? l1.val : 0;
        l1 = l1 != null ? l1.next : null;
        int n2 = l2 != null ? l2.val : 0;
        l2 = l2 != null ? l2.next : null;

        int sum = n1 + n2 + carry;
        ListNode node = new ListNode(sum % 10);
        carry = sum / 10;
        cur.next = node;
        cur = node;
    }

    return dummy.next;
}
\end{Code}

\newpage

\section{Add Two Numbers II} %%%%%%%%%%%%%%%%%%%%%%



\subsubsection{Description}
You are given two non-empty linked lists representing two non-negative integers. The most significant digit comes first and each of their nodes contain a single digit. Add the two numbers and return it as a linked list.

You may assume the two numbers do not contain any leading zero, except the number 0 itself.

\textbf{Follow up:}

What if you cannot modify the input lists? In other words, reversing the lists is not allowed.

\textbf{Example:}

\textbf{Input:} \code{(7 -> 2 -> 4 -> 3) + (5 -> 6 -> 4)}

\textbf{Output:} \code{7 -> 8 -> 0 -> 7}
\subsubsection{Solution}

\begin{Code}
public ListNode addTwoNumbers(ListNode l1, ListNode l2) {
    Stack<Integer> s1 = new Stack<Integer>();
    Stack<Integer> s2 = new Stack<Integer>();

    while (l1 != null) {
        s1.push(l1.val);
        l1 = l1.next;
    }
    ;
    while (l2 != null) {
        s2.push(l2.val);
        l2 = l2.next;
    }

    int sum = 0;
    ListNode list = new ListNode(0);
    while (!s1.empty() || !s2.empty()) {
        if (!s1.empty()) sum += s1.pop();
        if (!s2.empty()) sum += s2.pop();
        list.val = sum % 10;
        ListNode head = new ListNode(sum / 10);
        head.next = list;
        list = head;
        sum /= 10;
    }

    return list.val == 0 ? list.next : list;
}
\end{Code}

\newpage

\section{Reverse Linked List} %%%%%%%%%%%%%%%%%%%%%%



\subsubsection{Description}
Reverse a singly linked list.

\textbf{Hint:}

A linked list can be reversed either iteratively or recursively. Could you implement both?

\subsubsection{Solution I}

\begin{Code}
public ListNode reverseList(ListNode head) {
    if (head == null || head.next == null) {
        return head;
    }
    ListNode next = head.next;
    ListNode newHead = reverseList(next);
    next.next = head;
    head.next = null;
    return newHead;
}
\end{Code}

\subsubsection{Solution II}
\begin{Code}
// 耗时0ms
public ListNode reverseList2(ListNode head) {
    ListNode dummy = new ListNode(0);

    for (ListNode p = head; p != null; ) {
        ListNode next = p.next;
        p.next = dummy.next;
        dummy.next = p;
        p = next;
    }

    return dummy.next;
}
\end{Code}

\newpage

\section{Reverse Linked List II} %%%%%%%%%%%%%%%%%%%%%%



\subsubsection{Description}
Reverse a linked list from position m to n. Do it in-place and in one-pass.

For example:

Given \code{1->2->3->4->5->NULL}, m = 2 and n = 4,

return \code{1->4->3->2->5->NULL}.

\textbf{Note:}

Given m, n satisfy the following condition:

1 ≤ m ≤ n ≤ length of list.

\subsubsection{Solution}

\begin{Code}
public ListNode reverseBetween(ListNode head, int m, int n) {
    if (head == null) return null;
    ListNode dummy = new ListNode(0);
    dummy.next = head;
    ListNode pre = dummy;
    for (int i = 0; i < m - 1; i++) pre = pre.next;

    ListNode start = pre.next;
    ListNode then = start.next;

    for (int i = 0; i < n - m; i++) {
        start.next = then.next;
        then.next = pre.next;
        pre.next = then;
        then = start.next;
    }

    return dummy.next;
}
\end{Code}

\newpage

\section{Sort List} %%%%%%%%%%%%%%%%%%%%%%



\subsubsection{Description}
Sort a linked list in O(n log n) time using constant space complexity.

\subsubsection{Solution}

\begin{Code}
public ListNode sortList(ListNode head) {
    if (head == null || head.next == null)
        return head;

    ListNode prev = null, slow = head, fast = head;

    while (fast != null && fast.next != null) {
        prev = slow;
        slow = slow.next;
        fast = fast.next.next;
    }

    prev.next = null;
    ListNode l1 = sortList(head);
    ListNode l2 = sortList(slow);
    return merge(l1, l2);
}

ListNode merge(ListNode l1, ListNode l2) {
    ListNode l = new ListNode(0), p = l;

    while (l1 != null && l2 != null) {
        if (l1.val < l2.val) {
            p.next = l1;
            l1 = l1.next;
        } else {
            p.next = l2;
            l2 = l2.next;
        }
        p = p.next;
    }

    if (l1 != null)
        p.next = l1;

    if (l2 != null)
        p.next = l2;

    return l.next;
}
\end{Code}

\newpage

\section{Linked List Cycle} %%%%%%%%%%%%%%%%%%%%%%



\subsubsection{Description}
Given a linked list, determine if it has a cycle in it.

\textbf{Follow up:}

Can you solve it without using extra space?

\subsubsection{Solution}

\begin{Code}
public boolean hasCycle(ListNode head) {
    if (head == null) {
        return false;
    }

    ListNode fast = head.next, slow = head;

    for ( ; fast != null && fast.next != null; fast = fast.next.next, slow = slow.next) {
        if (fast == slow) {
            return true;
        }
    }

    return false;
}
\end{Code}

\newpage

\section{Linked List Cycle II} %%%%%%%%%%%%%%%%%%%%%%



\subsubsection{Description}
Given a linked list, return the node where the cycle begins. If there is no cycle, return null.

\textbf{Note:} Do not modify the linked list.

\textbf{Follow up:}

Can you solve it without using extra space?

\subsubsection{Solution}

\begin{Code}
public ListNode detectCycle(ListNode head) {
    if (head == null || head.next == null) return null;

    ListNode firstp = head;
    ListNode secondp = head;
    boolean isCycle = false;

    while (firstp != null && secondp != null) {
        firstp = firstp.next;
        if (secondp.next == null) return null;
        secondp = secondp.next.next;
        if (firstp == secondp) {
            isCycle = true;
            break;
        }
    }

    if (!isCycle) return null;
    firstp = head;
    while (firstp != secondp) {
        firstp = firstp.next;
        secondp = secondp.next;
    }

    return firstp;
}
\end{Code}

\newpage

\section{Odd Even Linked List} %%%%%%%%%%%%%%%%%%%%%%



\subsubsection{Description}
Given a singly linked list, group all odd nodes together followed by the even nodes. Please note here we are talking about the node number and not the value in the nodes.

You should try to do it in place. The program should run in O(1) space complexity and O(nodes) time complexity.

\textbf{Example:}
\begin{Code}
Given 1->2->3->4->5->NULL,
return 1->3->5->2->4->NULL.
\end{Code}

\textbf{Note:}

The relative order inside both the even and odd groups should remain as it was in the input.

The first node is considered odd, the second node even and so on ...

\subsubsection{Solution}

\begin{Code}
public ListNode oddEvenList(ListNode head) {
    ListNode odd = new ListNode(0), pOdd = odd;
    ListNode even = new ListNode(0), pEven = even;

    int index = 1;
    for (ListNode p = head; p != null; p = p.next) {
        if ((index++ & 1) > 0) {
            pOdd.next = p;
            pOdd = pOdd.next;
        } else {
            pEven.next = p;
            pEven = pEven.next;
        }
    }

    pOdd.next = null;
    pEven.next = null;

    pOdd.next = even.next;
    return odd.next;
}
\end{Code}

\newpage

\section{Merge Two Sorted Lists} %%%%%%%%%%%%%%%%%%%%%%



\subsubsection{Description}
Merge two sorted linked lists and return it as a new list. The new list should be made by splicing together the nodes of the first two lists.

\subsubsection{Solution}

\begin{Code}
// 耗时15ms
public ListNode mergeTwoLists(ListNode l1, ListNode l2) {
    ListNode dummy = new ListNode(0);
    ListNode p = l1, q = l2, cur = dummy;
    for ( ; p != null && q != null; ) {
        if (p.val < q.val) {
            cur.next = p;
            p = p.next;
        } else {
            cur.next = q;
            q = q.next;
        }
        cur = cur.next;
    }
    cur.next = p != null ? p : q;
    return dummy.next;
}
\end{Code}

\newpage

\section{Intersection of Two Linked Lists} %%%%%%%%%%%%%%%%%%%%%%



\subsubsection{Description}
Write a program to find the node at which the intersection of two singly linked lists begins.

For example, the following two linked lists:
\begin{Code}
A:          a1 → a2
                   ↘
                     c1 → c2 → c3
                   ↗
B:     b1 → b2 → b3
\end{Code}

begin to intersect at node c1.


\textbf{Notes:}

If the two linked lists have no intersection at all, return null.

The linked lists must retain their original structure after the function returns.

You may assume there are no cycles anywhere in the entire linked structure.

Your code should preferably run in O(n) time and use only O(1) memory.
\subsubsection{Solution}

\begin{Code}
public ListNode getIntersectionNode(ListNode headA, ListNode headB) {
    int lenA = 0, lenB = 0;
    for (ListNode p = headA; p != null; p = p.next, lenA++);
    for (ListNode p = headB; p != null; p = p.next, lenB++);
    ListNode p = lenA > lenB ? headA : headB;
    ListNode q = lenA > lenB ? headB : headA;
    for (int i = 0; i < Math.abs(lenA - lenB); i++, p = p.next);
    for ( ; p != null && q != null; p = p.next, q = q.next) {
        if (p == q) {
            return p;
        }
    }
    return null;
}
\end{Code}

\newpage

\section{Copy List with Random Pointer} %%%%%%%%%%%%%%%%%%%%%%



\subsubsection{Description}
A linked list is given such that each node contains an additional random pointer which could point to any node in the list or null.

Return a deep copy of the list.
\subsubsection{Solution}

\begin{Code}
public RandomListNode copyRandomList(RandomListNode head) {
    for (RandomListNode node = head; node != null; ) {
        RandomListNode next = node.next;

        RandomListNode copy = new RandomListNode(node.label);
        copy.next = next;
        node.next = copy;
        node = next;
    }

    for (RandomListNode node = head; node != null; ) {
        node.next.random = node.random != null ? node.random.next : null;
        node = node.next.next;
    }

    RandomListNode dummy = new RandomListNode(0), cur = dummy;
    for (RandomListNode node = head; node != null; ) {
        cur.next = node.next;
        cur = cur.next;

        node.next = node.next.next;
        node = node.next;
    }

    return dummy.next;
}
\end{Code}

\newpage

\section{Merge k Sorted Lists} %%%%%%%%%%%%%%%%%%%%%%



\subsubsection{Description}
Merge k sorted linked lists and return it as one sorted list. Analyze and describe its complexity.

\subsubsection{Solution}

\begin{Code}
/**
 * 这里要注意lists中可能有node为null
 */
public ListNode mergeKLists(ListNode[] lists) {
    ListNode dummy = new ListNode(0), cur = dummy;

    PriorityQueue<ListNode> queue = new PriorityQueue<>(new Comparator<ListNode>() {
        @Override
        public int compare(ListNode node1, ListNode node2) {
            if (node1.val == node2.val) {
                return 0;
            } else if (node1.val < node2.val) {
                return -1;
            } else {
                return 1;
            }
        }
    });

    for (ListNode node : lists) {
        if (node != null) {
            queue.offer(node);
        }
    }

    while (!queue.isEmpty()) {
        ListNode node = queue.poll();
        cur.next = node;
        cur = cur.next;
        if (node.next != null) {
            queue.offer(node.next);
        }
    }

    return dummy.next;
}
\end{Code}

\newpage

\section{Palindrome Linked List} %%%%%%%%%%%%%%%%%%%%%%



\subsubsection{Description}
Given a singly linked list, determine if it is a palindrome.

\textbf{Follow up:}

Could you do it in O(n) time and O(1) space?

\subsubsection{Solution}

\begin{Code}
// 耗时2ms
public boolean isPalindrome(ListNode head) {
    ListNode slow = head, fast = head;
    while (fast != null && fast.next != null) {
        slow = slow.next;
        fast = fast.next.next;
    }
    fast = reverse(slow);
    /**
     * 注意退出条件是p1 != slow
     */
    for (ListNode p1 = head, p2 = fast; p1 != slow; p1 = p1.next, p2 = p2.next) {
        if (p1.val != p2.val) {
            return false;
        }
    }
    return true;
}

private ListNode reverse(ListNode node) {
    ListNode dummy = new ListNode(0), cur = dummy;
    while (node != null) {
        ListNode next = node.next;
        node.next = cur.next;
        cur.next = node;
        node = next;
    }
    return dummy.next;
}
\end{Code}

\newpage

\section{Insertion Sort List} %%%%%%%%%%%%%%%%%%%%%%



\subsubsection{Description}
Sort a linked list using insertion sort.

\subsubsection{Solution}

\begin{Code}
public ListNode insertionSortList(ListNode head) {
    if (head == null) {
        return head;
    }
    ListNode helper = new ListNode(0); //new starter of the sorted list
    ListNode cur = head; //the node will be inserted
    ListNode pre = helper; //insert node between pre and pre.next
    ListNode next = null; //the next node will be inserted
    //not the end of input list
    while (cur != null) {
        next = cur.next;
        //find the right place to insert
        while (pre.next != null && pre.next.val < cur.val) {
            pre = pre.next;
        }
        //insert between pre and pre.next
        cur.next = pre.next;
        pre.next = cur;
        pre = helper;
        cur = next;
    }
    return helper.next;
}
\end{Code}

\newpage

\section{Remove Nth Node From End of List} %%%%%%%%%%%%%%%%%%%%%%



\subsubsection{Description}
Given a linked list, remove the nth node from the end of list and return its head.

For example,

   Given linked list: \code{1->2->3->4->5}, and n = 2.

   After removing the second node from the end, the linked list becomes \code{1->2->3->5}.

\textbf{Note:}

Given n will always be valid.

Try to do this in one pass.

\subsubsection{Solution}

\begin{Code}
public ListNode removeNthFromEnd(ListNode head, int n) {
    if (head == null) {
        return null;
    }

    ListNode p = head;
    for (int i = 1; i < n; i++) {
        p = p.next;
    }

    ListNode dummy = new ListNode(-1);
    ListNode cur = dummy;
    cur.next = head;

    for ( ; p.next != null; p = p.next) {
        cur = cur.next;
    }

    cur.next = cur.next.next;

    return dummy.next;
}
\end{Code}

\newpage

\section{Reorder List} %%%%%%%%%%%%%%%%%%%%%%



\subsubsection{Description}
Given a singly linked list L: L0?L1?…?Ln-1?Ln,
reorder it to: L0?Ln?L1?Ln-1?L2?Ln-2?…

You must do this in-place without altering the nodes' values.

\textbf{For example,}

Given {1,2,3,4}, reorder it to {1,4,2,3}.

\subsubsection{Solution}

\begin{Code}
public void reorderList(ListNode head) {
    if (head == null || head.next == null) return;

    ListNode p1 = head;
    ListNode p2 = head;
    while (p2.next != null && p2.next.next != null) {
        p1 = p1.next;
        p2 = p2.next.next;
    }

    ListNode preMiddle = p1;
    ListNode preCurrent = p1.next;
    while (preCurrent.next != null) {
        ListNode current = preCurrent.next;
        preCurrent.next = current.next;
        current.next = preMiddle.next;
        preMiddle.next = current;
    }

    p1 = head;
    p2 = preMiddle.next;
    while (p1 != preMiddle) {
        preMiddle.next = p2.next;
        p2.next = p1.next;
        p1.next = p2;
        p1 = p2.next;
        p2 = preMiddle.next;
    }
}
\end{Code}

\newpage

\section{Swap Nodes in Pairs} %%%%%%%%%%%%%%%%%%%%%%



\subsubsection{Description}
Given a linked list, swap every two adjacent nodes and return its head.

For example,

Given \code{1->2->3->4}, you should return the list as \code{2->1->4->3}.

Your algorithm should use only constant space. You may not modify the values in the list, only nodes itself can be changed.

\subsubsection{Solution}

\begin{Code}
public ListNode swapPairs(ListNode head) {
    ListNode dummy = new ListNode(0);

    ListNode node = head, tail = dummy;

    for ( ; node != null && node.next != null; ) {
        ListNode next = node.next;
        node.next = tail.next;
        tail.next = node;

        ListNode nnext = next.next;
        next.next = node;
        tail.next = next;
        tail = node;

        node = nnext;
    }

    tail.next = node;

    return dummy.next;
}
\end{Code}

\newpage

\section{Remove Linked List Elements} %%%%%%%%%%%%%%%%%%%%%%



\subsubsection{Description}
Remove all elements from a linked list of integers that have value val.

\textbf{Example}

Given: \code{1 --> 2 --> 6 --> 3 --> 4 --> 5 --> 6, val = 6}

Return: \code{1 --> 2 --> 3 --> 4 --> 5}

\subsubsection{Solution}

\begin{Code}
public ListNode removeElements(ListNode head, int val) {
    ListNode dummy = new ListNode(0), node = dummy;
    for ( ; head != null; head = head.next) {
        if (head.val != val) {
            node.next = head;
            node = node.next;
        }
    }
    node.next = null;
    return dummy.next;
}
\end{Code}

\newpage

\section{Remove Duplicates from Sorted List} %%%%%%%%%%%%%%%%%%%%%%



\subsubsection{Description}
Given a sorted linked list, delete all duplicates such that each element appear only once.

For example,

Given \code{1->1->2}, return \code{1->2}.

Given \code{1->1->2->3->3}, return \code{1->2->3}.

\subsubsection{Solution}

\begin{Code}
public ListNode deleteDuplicates(ListNode head) {
    ListNode dummy = new ListNode(0), cur = dummy;
    for ( ; head != null; head = head.next) {
        if (cur == dummy || head.val != cur.val) {
            cur.next = head;
            cur = cur.next;
        }
    }
    cur.next = null;
    return dummy.next;
}
\end{Code}

\newpage

\section{Remove Duplicates from Sorted List II} %%%%%%%%%%%%%%%%%%%%%%



\subsubsection{Description}
Given a sorted linked list, delete all nodes that have duplicate numbers, leaving only distinct numbers from the original list.

For example,

Given \code{1->2->3->3->4->4->5}, return \code{1->2->5}.
Given \code{1->1->1->2->3}, return \code{2->3}.

\subsubsection{Solution}

\begin{Code}
public ListNode deleteDuplicates(ListNode head) {
    if (head == null) {
        return null;
    }

    ListNode dummy = new ListNode(0), tail = dummy;
    ListNode prev = head, cur = head.next;

    for ( ; cur != null; cur = cur.next) {
        if (prev.val != cur.val) {
            if (prev.next == cur) {
                tail.next = prev;
                tail = tail.next;
            }
            prev = cur;
        }
    }

    tail.next = prev.next == null ? prev : null;
    return dummy.next;
}
\end{Code}

\newpage

\section{Convert Sorted List to Binary Search Tree} %%%%%%%%%%%%%%%%%%%%%%



\subsubsection{Description}
Given a singly linked list where elements are sorted in ascending order, convert it to a height balanced BST.
\subsubsection{Solution}

\begin{Code}
public TreeNode sortedListToBST(ListNode head) {
    if (head == null) return null;
    return toBST(head, null);
}

public TreeNode toBST(ListNode head, ListNode tail) {
    ListNode slow = head;
    ListNode fast = head;
    if (head == tail) return null;

    while (fast != tail && fast.next != tail) {
        fast = fast.next.next;
        slow = slow.next;
    }
    TreeNode thead = new TreeNode(slow.val);
    thead.left = toBST(head, slow);
    thead.right = toBST(slow.next, tail);
    return thead;
}
\end{Code}

\newpage

\section{Partition List} %%%%%%%%%%%%%%%%%%%%%%



\subsubsection{Description}
Given a linked list and a value x, partition it such that all nodes less than x come before nodes greater than or equal to x.

You should preserve the original relative order of the nodes in each of the two partitions.

For example,

Given \code{1->4->3->2->5->2} and x = 3,

return \code{1->2->2->4->3->5}.

\subsubsection{Solution}

\begin{Code}
ListNode partition(ListNode head, int x) {
    ListNode node1 = new ListNode(0);
    ListNode node2 = new ListNode(0);
    ListNode p1 = node1, p2 = node2;

    while (head != null) {
        if (head.val < x)
            p1 = p1.next = head;
        else
            p2 = p2.next = head;
        head = head.next;
    }
    p2.next = null;
    p1.next = node2.next;
    return node1.next;
}
\end{Code}

\newpage

\section{Reverse Nodes in k-Group} %%%%%%%%%%%%%%%%%%%%%%



\subsubsection{Description}
Given a linked list, reverse the nodes of a linked list k at a time and return its modified list.

k is a positive integer and is less than or equal to the length of the linked list. If the number of nodes is not a multiple of k then left-out nodes in the end should remain as it is.

You may not alter the values in the nodes, only nodes itself may be changed.

Only constant memory is allowed.

For example,

Given this linked list: \code{1->2->3->4->5}

For k = 2, you should return: \code{2->1->4->3->5}

For k = 3, you should return: \code{3->2->1->4->5}

\subsubsection{Solution}

\begin{Code}
public ListNode reverseKGroup(ListNode head, int k) {
    int size = 0;
    ListNode dummy = new ListNode(0), cur = dummy, p;
    for (p = head; p != null; p = p.next, size++);

    for (p = head; size >= k; size -= k) {
        ListNode tail = p;
        for (int i = 0; i < k; i++) {
            ListNode next = p.next;
            p.next = cur.next;
            cur.next = p;
            p = next;
        }
        cur = tail;
    }
    cur.next = p;
    return dummy.next;
}
\end{Code}

\newpage

\section{Rotate List} %%%%%%%%%%%%%%%%%%%%%%



\subsubsection{Description}
Given a list, rotate the list to the right by k places, where k is non-negative.

For example:

Given \code{1->2->3->4->5->NULL} and k = 2,
return \code{4->5->1->2->3->NULL}.

\subsubsection{Solution}

\begin{Code}
public ListNode rotateRight(ListNode head, int n) {
    if (head == null || head.next == null) return head;
    ListNode dummy = new ListNode(0);
    dummy.next = head;
    ListNode fast = dummy, slow = dummy;

    int i;
    for (i = 0; fast.next != null; i++)//Get the total length
        fast = fast.next;

    for (int j = i - n % i; j > 0; j--) //Get the i-n%i th node
        slow = slow.next;

    fast.next = dummy.next; //Do the rotation
    dummy.next = slow.next;
    slow.next = null;

    return dummy.next;
}
\end{Code}

\newpage

\section{Plus One Linked List} %%%%%%%%%%%%%%%%%%%%%%



\subsubsection{Description}
Given a non-negative integer represented as non-empty a singly linked list of digits, plus one to the integer.

You may assume the integer do not contain any leading zero, except the number 0 itself.

The digits are stored such that the most significant digit is at the head of the list.

\textbf{Example:}
\begin{Code}
Input:
1->2->3

Output:
1->2->4
\end{Code}

\subsubsection{Solution}

\begin{Code}
public ListNode plusOne(ListNode head) {
    if (head == null) {
        return head;
    }
    Stack<ListNode> stack = new Stack<ListNode>();
    for (ListNode node = head; node != null; node = node.next) {
        stack.push(node);
    }
    int k = 1;
    while (!stack.isEmpty()) {
        ListNode node = stack.pop();
        int val = node.val + k;
        node.val = val % 10;
        k = val / 10;
        if (k == 0) {
            return head;
        }
    }
    ListNode node = new ListNode(k);
    node.next = head;
    return node;
}
\end{Code}

\newpage

\section{Min Stack} %%%%%%%%%%%%%%%%%%%%%%



\subsubsection{Description}
Design a stack that supports push, pop, top, and retrieving the minimum element in constant time.
\begin{Code}
push(x) -- Push element x onto stack.
pop() -- Removes the element on top of the stack.
top() -- Get the top element.
getMin() -- Retrieve the minimum element in the stack.
\end{Code}

\textbf{Example:}
\begin{Code}
MinStack minStack = new MinStack();
minStack.push(-2);
minStack.push(0);
minStack.push(-3);
minStack.getMin();   --> Returns -3.
minStack.pop();
minStack.top();      --> Returns 0.
minStack.getMin();   --> Returns -2.
\end{Code}
\subsubsection{Solution}

\begin{Code}
Stack<Integer> mMinStack;

Stack<Integer> mStack;

public MinStack() {
    mStack = new Stack<Integer>();
    mMinStack = new Stack<Integer>();
}

public void push(int x) {
    mStack.push(x);

    // 注意这里要判空
    if (mMinStack.isEmpty() || x < mMinStack.peek()) {
        mMinStack.push(x);
    } else {
        mMinStack.push(mMinStack.peek());
    }
}

public void pop() {
    mStack.pop();
    mMinStack.pop();
}

public int top() {
    return mStack.peek();
}

public int getMin() {
    return mMinStack.peek();
}
\end{Code}

\newpage

\section{Evaluate Reverse Polish Notation} %%%%%%%%%%%%%%%%%%%%%%



\subsubsection{Description}
Evaluate the value of an arithmetic expression in Reverse Polish Notation.

Valid operators are \code{+, -, *, /}. Each operand may be an integer or another expression.

Some examples:
\begin{Code}
  ["2", "1", "+", "3", "*"] -> ((2 + 1) * 3) -> 9
  ["4", "13", "5", "/", "+"] -> (4 + (13 / 5)) -> 6
\end{Code}

\subsubsection{Solution}

\begin{Code}
public int evalRPN(String[] tokens) {
    int a, b;
    Stack<Integer> S = new Stack<Integer>();
    for (String s : tokens) {
        if (s.equals("+")) {
            S.add(S.pop() + S.pop());
        } else if (s.equals("/")) {
            b = S.pop();
            a = S.pop();
            S.add(a / b);
        } else if (s.equals("*")) {
            S.add(S.pop() * S.pop());
        } else if (s.equals("-")) {
            b = S.pop();
            a = S.pop();
            S.add(a - b);
        } else {
            S.add(Integer.parseInt(s));
        }
    }
    return S.pop();
}
\end{Code}

\newpage

\section{Basic Calculator} %%%%%%%%%%%%%%%%%%%%%%



\subsubsection{Description}
Implement a basic calculator to evaluate a simple expression string.

The expression string may contain open ( and closing parentheses ), the plus + or minus sign -, non-negative integers and empty spaces .

You may assume that the given expression is always valid.

Some examples:
\begin{Code}
"1 + 1" = 2
" 2-1 + 2 " = 3
"(1+(4+5+2)-3)+(6+8)" = 23

\textbf{Note:}

Do not use the eval built-in library function.

\end{Code}

\subsubsection{Solution}

\begin{Code}
public int calculate(String s) {
    Stack<Integer> stack = new Stack<>();
    int result = 0;
    int number = 0;
    int sign = 1;
    for (int i = 0; i < s.length(); i++) {
        char c = s.charAt(i);
        if (Character.isDigit(c)) {
            number = 10 * number + (int) (c - '0');
        } else if (c == '+') {
            result += sign * number;
            number = 0;
            sign = 1;
        } else if (c == '-') {
            result += sign * number;
            number = 0;
            sign = -1;
        } else if (c == '(') {
            //we push the result first, then sign;
            stack.push(result);
            stack.push(sign);
            //reset the sign and result for the value in the parenthesis
            sign = 1;
            result = 0;
        } else if (c == ')') {
            result += sign * number;
            number = 0;
            result *= stack.pop();    //stack.pop() is the sign before the parenthesis
            result += stack.pop();   //stack.pop() now is the result calculated before the parenthesis

        }
    }
    if (number != 0) result += sign * number;
    return result;
}
\end{Code}

\newpage

\section{Remove Duplicate Letters} %%%%%%%%%%%%%%%%%%%%%%



\subsubsection{Description}
Given a string which contains only lowercase letters, remove duplicate letters so that every letter appear once and only once. You must make sure your result is the smallest in lexicographical order among all possible results.

\textbf{Example:}

Given "bcabc"

Return "abc"

Given "cbacdcbc"

Return "acdb"

\subsubsection{Solution}

\begin{Code}
public String removeDuplicateLetters(String s) {
    if (s.length() == 0) {
        return "";
    }

    int[] f = new int[26];
    for (char c : s.toCharArray()) {
        f[c - 'a']++;
    }

    int pos = 0;
    /**
     * 不断尽可能往后走,直到要略过唯一剩下的那个字符时停下
     */
    for (int i = 0; i < s.length(); i++) {
        /**
         * 这里记录下最小的那个字符最开始出现的位置,为什么不记录该字符别的位置呢,比如"abacb",如果这里取最后一个a的位置,最后结果会是acb,但应该是abc。
         */
        if (s.charAt(i) < s.charAt(pos)) {
            pos = i;
        }
        /**
         * 这里因为要略过当前字符了,所以剩余的字符串里这个字符数要减1,如果为0说明这个字符
         * 只剩唯一一个了,不能再往后走了
         */
        if (--f[s.charAt(i) - 'a'] == 0) {
            break;
        }
    }

    String right = s.substring(pos + 1).replaceAll(s.substring(pos, pos + 1), "");
    return s.charAt(pos) + removeDuplicateLetters(right);
}
\end{Code}

\newpage

\section{Implement Queue using Stacks} %%%%%%%%%%%%%%%%%%%%%%



\subsubsection{Description}
Implement the following operations of a queue using stacks.
\begin{Code}
push(x) -- Push element x to the back of queue.
pop() -- Removes the element from in front of queue.
peek() -- Get the front element.
empty() -- Return whether the queue is empty.
\end{Code}

\textbf{Notes:}

You must use only standard operations of a stack -- which means only push to top, peek/pop from top, size, and is empty operations are valid.

Depending on your language, stack may not be supported natively. You may simulate a stack by using a list or deque (double-ended queue), as long as you use only standard operations of a stack.

You may assume that all operations are valid (for example, no pop or peek operations will be called on an empty queue).

\subsubsection{Solution}

\begin{Code}
public class MyQueue {

    private Stack<Integer> mStack = new Stack<Integer>();
    private Stack<Integer> mStackTmp = new Stack<Integer>();

    // Push element x to the back of queue.
    public void push(int x) {
        mStack.push(x);
    }

    // Removes the element from in front of queue.
    public void pop() {
        dump(mStack, mStackTmp);
        mStackTmp.pop();
        dump(mStackTmp, mStack);
    }

    // Get the front element.
    public int peek() {
        dump(mStack, mStackTmp);
        int peek = mStackTmp.peek();
        dump(mStackTmp, mStack);
        return peek;
    }

    // Return whether the queue is empty.
    public boolean empty() {
        return mStack.isEmpty();
    }

    private void dump(Stack<Integer> left, Stack<Integer> right) {
        while (!left.isEmpty()) {
            right.push(left.pop());
        }
    }
}

\end{Code}

\newpage

\section{Flatten Nested List Iterator} %%%%%%%%%%%%%%%%%%%%%%



\subsubsection{Description}
Given a nested list of integers, implement an iterator to flatten it.

Each element is either an integer, or a list -- whose elements may also be integers or other lists.

\textbf{Example 1:}

Given the list [[1,1],2,[1,1]],

By calling next repeatedly until hasNext returns false, the order of elements returned by next should be: [1,1,2,1,1].

\textbf{Example 2:}

Given the list [1,[4,[6]]],

By calling next repeatedly until hasNext returns false, the order of elements returned by next should be: [1,4,6].

\subsubsection{Solution}

\begin{Code}
public abstract class NestedIterator implements Iterator<Integer> {

    private Stack<NestedInteger> stack;

    public NestedIterator(List<NestedInteger> nestedList) {
        stack = new Stack<NestedInteger>();
        push(nestedList);
    }

    private void push(List<NestedInteger> nestedList) {
        for (int i = nestedList.size() - 1; i >= 0; i--) {
            NestedInteger nest = nestedList.get(i);
            if (nest.isInteger()) {
                stack.push(nest);
            } else {
                push(nest.getList());
            }
        }
    }

    @Override
    public Integer next() {
        return stack.pop().getInteger();
    }

    @Override
    public boolean hasNext() {
        return !stack.isEmpty();
    }
}
\end{Code}

\newpage

\section{Implement Stack using Queues} %%%%%%%%%%%%%%%%%%%%%%



\subsubsection{Description}
Implement the following operations of a stack using queues.
\begin{Code}
push(x) -- Push element x onto stack.
pop() -- Removes the element on top of the stack.
top() -- Get the top element.
empty() -- Return whether the stack is empty.
\end{Code}

\textbf{Notes:}

You must use only standard operations of a queue -- which means only push to back, peek/pop from front, size, and is empty operations are valid.

Depending on your language, queue may not be supported natively. You may simulate a queue by using a list or deque (double-ended queue), as long as you use only standard operations of a queue.

You may assume that all operations are valid (for example, no pop or top operations will be called on an empty stack).

\subsubsection{Solution}

\begin{Code}

\end{Code}

\newpage

\section{Title} %%%%%%%%%%%%%%%%%%%%%%



\subsubsection{Description}

\subsubsection{Solution}

\begin{Code}

\end{Code}

\newpage

